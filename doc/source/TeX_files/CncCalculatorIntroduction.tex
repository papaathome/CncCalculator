\chapter{Introduction}\label{Introduction}

Getting access to a CNC machine has become more easy over the years.
Using such machines optimal requires some a propper configuration, a choice of the right tools
and to do some calculations. The questions behind the most bacic calculations are `How fast should
my spindle turn?' and `How fast can I move through the material?'

The calculations involved are quite simple, with the correct information just multiply and divide
a few values. It becomes a bit more ticky when some of the values involved are given in
the metric sysem and others are in the imperial system. Conversion between the unit systems is
an area where a mistake is made easily but often spotted too late.

\CC\ will do all the calculations if provided with the right information and will do
unit conversion `on the fly' as required. In this document the calculations used are explained
as wel as the method of conversion between the unit systems.

\CC\ being what it is, a tool that helps the user, will provide you with a calculated
result. It is still up to you to make an estimate if this is a reasonable result. For this it
helps to have at least a look at the calculations and also on the internet for similar applications
of tools and material to get a feeling of what a reasonable result is. In this practice is the
teacher for good results and no tool can help you with that.

This document assumes that you have some basic knowledge on what is involved in the physical process
of cutting a part from some stock and know how to look up thinks in your machine documentation or
on the internet. The actual process of cutting material is explained in short but not in full details.
This is not a introduction on cutting done by a CNC machine but an explanation on how \CC\ works.
