\chapter{Quick start}\label{QuickStart}

\CC\ is a tool to quickly calculate feeds and speeds born out of necessity. Too often I made a mistake
in the calculations (or conversion) involved. ON the internet (or in the store for your smart phone)
more than one tool can be found for doing the calculations but I never found one I was happy with.
I decided to make one so I don't have to look for it any more, can change it to whatever I want it to be,
use for free and without any spam involved.

\CC\ has some strong point and weak points if you compare it with the other ones available.
A few of the points are listed below. You can classify it as a strong or weak point for yourself.
\begin{itemize}
    \item All source code is available and under a permissive licence form.
    \item Selection of units, e.g.: [in] or [mm].
    \item Conversion of units, e.g.: from [in/min] to [mm/sec].
    \item Includes a list of materials with preset values.
    \item Includes a list of tools with preset values.
    \item Can import any FreeCAD tools library (both v0.18 or before and v0.19 or better)
    \item Where you put it on the screen there it will be the next time you start it.
    \item Easy and intuitive to use.
\end{itemize}

\section{A simple introduction}
Assuming that you know what `Feeds' and `Speeds' are this is a quick run trough for \CC.

When you start \CC\ for the first time it comes up on a default location and using default settings.

You can drag the form to any location and when you close the application it wil reopen on exact
the location where you closed it. This is handy when you arrange your (CNC related) tools on fixed
positions on the screen.

By default, on the top you find a menu bar with entries like `file' and `about', below that is an
tools bar. You can drag this tools bar to any side of the application, top, left, right or bottom.
The tools bar contains two selection buttons, one for a tool and another for a material. Selecting
one will fill in the `Tool' or `Material' section with values from the selected item. You do not
have to select any tool or material, it is possible to use \CC\ by filling in all required information
by typing data in the `Tool'  and `Material' section. (see below)

There are two sections for input, `tool' and `material', and one for result. Working with wood it
is common to use a recomended spindle speed, working with other materials a more accurate specification
of the tool cutting speed is required. Selecting between the two methods is done by selecting the
round selection button after `Cutting speed' or `Spindle speed' in the material section. The one
not selected will be calculated from the other and a given tool diameter.

Input fields are coloured white\footnote{Depending on system settings.} and calculated fields are gray.
Next to the field with the value is a button with a list of units. For edit fields you can select
the appropriate units for your value. No `on the fly' conversion is done, assuming that the value
is the right value for the units you select. For calculated fields an `on the fly' conversion is
done and the value is changed to the correct value for the units you selected.

You can select any value calculated or any value in an edit field and use `copy/paste' shortcuts
to get the values to/from the clip board. Or you could just read and type it from/into your final
application.
